\documentclass{beamer}
\usepackage[english,italian]{babel}

\usepackage[T1]{fontenc}
\usepackage{textcomp}
\usepackage[utf8]{inputenc}	%caratteri accentati

\newcommand\tab[1][1cm]{\hspace*{#1}}	% define new command tab

\usepackage[font=scriptsize]{subcaption}
\usepackage[most]{tcolorbox}

%insert code
\usepackage{listings}

%------------------------------------------------
%		Trasparenza liste
%------------------------------------------------
\setbeamercovered{transparent}


%------------------------------------------------
%		IMAGE DEFINITIONS
%------------------------------------------------
\usepackage{changepage}

\usepackage{pgf,tikz}
\usepackage{graphicx}

\pgfdeclareimage[height=4cm]{GraphExample}{./images/Graph/GraphExample}
\pgfdeclareimage[height=4cm]{socialNetwork}{./images/Graph/socialNetwork}
\pgfdeclareimage[height=4cm]{SIRGraphFacebook}{./images/Graph/SIRGraphFacebook}
\pgfdeclareimage[height=3.5cm]{sirModel}{./images/SIR/sir}
\pgfdeclareimage[height=2.5cm]{seirModel}{./images/SIR/SEIR-SEIRS}
\pgfdeclareimage[height=2.5cm]{sisModel}{./images/SIR/SI-SIS}
\pgfdeclareimage[height=2.5cm]{sirsModel}{./images/SIR/SIR-SIRS}

\pgfdeclareimage[width=12.8cm]{randomFunction}{./images/Code/RandomFunction}
\pgfdeclareimage[width=12.8cm]{preferentialAttachmentFunction}{./images/Code/preferentialAttachmentFunction}


\pgfdeclareimage[height=4cm]{DegreeHistogramRandom}{./images/Graph/DegreeHistogramRandom}
\pgfdeclareimage[width=6cm]{poissonDistribution}{./images/Graph/poissonDistribution}
\pgfdeclareimage[height=4cm]{DegreeHistogramPrefAttach01}{./images/Graph/DegreeHistogramPrefAttach}
\pgfdeclareimage[height=4cm]{DegreeHistogramPrefAttach02}{./images/Graph/DegreeHistogramPrefAttach01}

\pgfdeclareimage[height=6cm]{PANet}{./images/Graph/PAa0.25m4.5}
\pgfdeclareimage[height=6cm]{RNet}{./images/Graph/RNet}
\pgfdeclareimage[height=6cm]{FNet}{./images/Graph/FNet}
\pgfdeclareimage[height=6cm]{PASIR}{./images/Graph/PASIR}
\pgfdeclareimage[height=6cm]{RSIR}{./images/Graph/RSIR}
\pgfdeclareimage[height=6cm]{FSIR}{./images/Graph/FSIR}


\definecolor{blueHKBU}{RGB}{0,0,210}

\usetheme{Padova}

\title{\vspace{1cm}How the spread of disease change on synthetic network and real network, simulation and comparison.}
\subtitle{\vspace{-0cm}COMP7960 Research I}
\author{%\vspace{-0.5cm}\textbf{Laureando}:\\
	Matteo Azzarelli}
\date{\vspace{2.5cm}\color{black}{\footnotesize{Academic year 2018/2019}}}



\begin{document}

    \AtBeginSection[]{
      \begin{frame}{~}
      \vfill
      \centering
      \begin{tcolorbox}[halign=center,colframe=blueHKBU, colback=blueHKBU!5!white]
        \usebeamerfont{title}\insertsectionhead\par%
      \end{tcolorbox}
      \vfill
      \end{frame}
    }
    
	\maketitle

	\begin{frame}{Outline}
		\tableofcontents
	\end{frame}

	\section{Introduction}
	
	\begin{frame}{Introduction}
		Why we developed this tool?
		\pause
		\begin{enumerate}[<+->]
			\item Evaluate the different spread behaviours of disease on different network using Python.
			\item Analyse how the spread of disease change based on different network connections using \textbf{SIR model}.
			\item Compare synthetic network with real network, Facebook like.
			
		\end{enumerate}
	\end{frame}

	\section{Graph - Network}
	
	\begin{frame}{Network}
		What is it a \textbf{Network}?
		
		\vspace{0.5cm}
		A \textbf{Network} is a particular representation of a \textit{Graph}.
		
		\vspace{0.5cm}
		A Graph is a pair of set, $G = (V, E)$, where 
		\begin{itemize}
		    \item \textbf{V} is the set of \textit{vertices} or also called \textit{
nodes}.
		    \item \textbf{E} is a set of \textit{edges}. An edge is a link between two Vertices.
		\end{itemize}
		\begin{center}
			\pgfuseimage{GraphExample}
		\end{center}
	\end{frame}
	

	\begin{frame}{Network physical meaning}
	    The fiscal meaning is:
		\begin{itemize}[<+->]
		    \item \textbf{Node}: represent the pool of person. So, each node correspond to a person in this network.
		    \item \textbf{Edges}: represent communications among people, so if there is a link between two nodes that means that they are in contact.
		    \item \textbf{Network}: is a portion of the real world that represent the connection and interaction in this mini-world.
		\end{itemize}
		\begin{left}
			\pgfuseimage{socialNetwork}
		\end{left}
	\end{frame}



	\section{SIR model}
	
	\begin{frame}{Compartmental models}
	    \textbf{Compartmental models} are a technique used to simplify the mathematical modelling of infectious disease.
	    
	    The \textbf{population} is \textbf{divided} into compartments, with the assumption that every individual in the same compartment has the \textbf{same characteristics}.
	\end{frame}
	
	\begin{frame}{SIR model}
	    What is it SIR model?
	    \begin{center}
			\pgfuseimage{sirModel}
		\end{center}
		\begin{itemize}\footnotesize
		    \item \textbf{Susceptible}: it means that this person is susceptible to ketch the disease.
		    \item \textbf{Infected}: this status represent people that have the disease and that can spread it.
		    \item \textbf{Recovered}: represent the people that had the disease and now they are recovered/immunised.
		\end{itemize}
	\end{frame}
	
	\begin{frame}{More models}
	    There are many variants of this model like:
        \begin{itemize}[<+->]
            \item \textbf{SEIRS}\only<1>{: this model provide the Exposed(E) class where the person has the disease but it doesn’t spread it on to susceptible nodes.} \only<1,4>{This kind of model is suitable for disease with a \textbf{long incubation disease}.}\only<1>{
            \begin{center}
			    \pgfuseimage{seirModel}
		    \end{center}}
            \item \textbf{SIS}\only<2>{: here there are only three states and modelling disease that after the state of infection the people come back to the state of susceptible.
            \begin{center}
			    \pgfuseimage{sisModel}
		    \end{center}
            }
            \item \textbf{SIRS}\only<3>{: this one is a variant of the previous model that introduce a stage of immunity for a certain time after that the person can be infected again.
            \begin{center}
			    \pgfuseimage{sirsModel}
		    \end{center}
            }
        \end{itemize}

        \only<4>{The previous last two models are typically used to represent different \textbf{epidemics disease} such as HIV or SARS [3].}
	\end{frame}
	
% 	\begin{frame}{Plagio}
% 		\begin{alertblock}{Definizione}
% 			Un programma che è stato prodotto da un altro e riportato con un numero esiguo di trasformazioni di routine.
			
% 			\hspace{4cm}\textit{1976. Alan Parker e James O. Hamblen.}
% 		\end{alertblock}
		
% 		\vspace{-0.5cm}
% 		\begin{center}
% 			\pgfuseimage{PlagiarismLevels}
% 		\end{center}
% 	\end{frame}
	

	\section{Network Models}
	
% 	\begin{frame}{Studio ed Implementazione}
% 		Caratteristiche dell'applicazione:
% 		\begin{itemize}
% 			\item<1-> Sviluppata in \textbf{Java}:
% 				\begin{itemize}
% 					\item \`E cross-platform.
% 					\item Documentazione completa.
% 					\item Esistenza di librerie per il nostro scopo.
% 				\end{itemize}
% 			\item<2-> Utilizzo del pattern Model View Controller.
% 			\item<3-> Download delle repositories di GitHub.
% 			\item<4-> Analisi dei progetti degli studenti tramite il servizio MOSS.
% 		\end{itemize}
% 	\end{frame}
	
	\begin{frame}{Python}
		We chose \textbf{Python}:
		\begin{itemize}
		    \item Easy and flexible language.
		    \item There are some libraries that help us like 
		    \begin{itemize}
		        \item \textbf{NetworkX}: it provide a good \textbf{representation for graphs} and networks. In addition, it gives many related algorithm.
		        \item \textbf{MatPlotLib}: to draw and print into files our graphs.
		    \end{itemize}
		\end{itemize}
	\end{frame}
	
	\begin{frame}{SIR parameter}
	    \begin{alertblock}{Definition of Neighbour}
			The set of neighbour of a vertex $v \in V$ is denoted with $N_G(v) = \{u_1, u_2, ..., u_i \}$, where $u_k,~k=1,...,i$, is adjacent to a given vertex \verb|v|.
		\end{alertblock}
		\pause
	    \begin{itemize}
	        \item \textbf{$\beta$}: the susceptible population have a fixed probability per unit time to get the disease from any infected neighbour.
	        \item \textbf{$\gamma$}: the fixed probability per unit time that represent the probability to be recovered/immunised.
	    \end{itemize}
	\end{frame}
	
	\begin{frame}{SIR equation}
	    With this parameter we can model the population susceptible, infected and recovered by a system of ordinary differential equation:
	    \begin{equation}\label{eq:diffEq}
          \left\{
            \begin{matrix}
                \frac{\mathrm{d} s}{\mathrm{d} t} = -\beta si, &&
                \frac{\mathrm{d} i}{\mathrm{d} t} = \beta si - \gamma i, &&
                \frac{\mathrm{d} r}{\mathrm{d} t} = \gamma i
            \end{matrix}\right.
        \end{equation}
        \begin{center}
			\pgfuseimage{SIRGraphFacebook}
		\end{center}
		
		For our simulation we used $\beta = 0.03$ and $\gamma = 0.04$ .
	\end{frame}
	
	\begin{frame}{Random Network Function}
        \begin{adjustwidth}{-1cm}{-1cm}
			\pgfuseimage{randomFunction}
		\end{adjustwidth}
	\end{frame}
	
	\begin{frame}{R Degree distribution}
		Because of the nature of random number, the nodes' degree distribution will follow the Poisson distribution, defined by the Eq.\ref{eq:poisson}.
        \begin{equation}\label{eq:poisson}
          p(k) \approx \frac{\lambda^ke^{-\lambda}}{k!}
        \end{equation}
        The parameter $\lambda = N P$, where $N = |V|$ and P is the probability to establish a connection. 
        
        \begin{left}
			\pgfuseimage{DegreeHistogramRandom}
			\pgfuseimage{poissonDistribution}
		\end{left}
	\end{frame}
	\begin{frame}{Random Network}
	    \begin{center}
			\pgfuseimage{RNet}
		\end{center}
	    \footnotesize{
		    N = 1000,
		    P = 0.052
		}
	\end{frame}
	
	\begin{frame}{Preferential attachment Network Function}
        \begin{adjustwidth}{-1cm}{-1cm}
			\pgfuseimage{preferentialAttachmentFunction}
		\end{adjustwidth}
	\end{frame}
	
	\begin{frame}{Preferential attachment}
		The probability if is defined by the preferential attachment equation Eq.\ref{eq:prefAttach}.
        \begin{equation}\label{eq:prefAttach}
          P(k) = \frac{B(k+a,\gamma)}{B(K_0 + a, \gamma - 1)}
        \end{equation}
        
        for $k \geq k_0$ and zero otherwise, where $B(x,y)$ is the Euler beta function Eq.\ref{eq:euler}:
        \begin{equation}\label{eq:euler}
          B(x,y) = \frac{\Gamma(x)\Gamma(y)}{\Gamma(x+y)}
        \end{equation}
        
        with $\Gamma(x)$ is the standard gamma function, and $\gamma$ is defined as in the  Eq.\ref{eq:gamma}
        \begin{equation}\label{eq:gamma}
          B(x,y) = 2+ \frac{k_0+a}{m}
        \end{equation}
        
        \hfill{continue...}
        
    \end{frame}
    
    \begin{frame}{PA Degree distribution}
        \textbf{a}~$> -k_0$ characterise the \textbf{width} of the curve,
        
        \textbf{m} is the \textbf{number of link} that will \textbf{added}, so the slope of this curve
        
        \begin{left}
			\pgfuseimage{DegreeHistogramPrefAttach01}
			\pgfuseimage{DegreeHistogramPrefAttach02}
		\end{left}
		
		\footnotesize{
		    N = 1000,
		    a = 0.25,
		    m = 4.5
		}
		\hfill
		\footnotesize{
		    N = 1000,
		    a = 50,
		    m = 0.1
		}
	\end{frame}
	
	\begin{frame}{Preferential Attachment Network}
	    \begin{center}
			\pgfuseimage{PANet}
		\end{center}
	    \footnotesize{
		    N = 1000,
		    a = 0.25,
		    m = 4.5
		}
	\end{frame}
	
	\begin{frame}{Facebook Network}
	    Data set taken from Stanford SNAP, it represent a portion of Facebook network.
	    \begin{center}
			\pgfuseimage{FNet}
		\end{center}
	    \footnotesize{
		    N = 4039,
		    E = 88234
		}
	\end{frame}
	
	\section{Results}
	
	\begin{frame}{SIR Comparison}
	    For our test we set the parameter $|V|=N=1000$, $\beta = 0.01$ (transmission rate) and $\gamma= 0.04$ (recovery rate).
    
        Degree of our network:
        \begin{itemize}
            \item Synthetic network, Random and Preferential Attachment Degree around 20 to 50 link per node.
            \item Facebook network degree is 175 link per node
        \end{itemize}
	\end{frame}
	
	\begin{frame}{Random Network SIR}
	    \begin{center}
			\pgfuseimage{RSIR}
		\end{center}
		\footnotesize{
		    N = 1000,
		    P = 0.052,
		    $\beta = 0.01$ and $\gamma= 0.04$
		}
	\end{frame}
	
	\begin{frame}{Preferential Attachment Network SIR}
	    \begin{center}
			\pgfuseimage{PASIR}
		\end{center}
		\footnotesize{
		    N = 1000,
		    a = 0.25,
		    m = 4.5,
		    $\beta = 0.01$ and $\gamma= 0.04$
		}
	\end{frame}
	
	\begin{frame}{Facebook Network SIR}
	    \begin{center}
			\pgfuseimage{FSIR}
		\end{center}
		\footnotesize{
		    N = 4039,
		    E = 88234,
		    $\beta = 0.01$ and $\gamma= 0.04$
		}
	\end{frame}
	
	\begin{frame}{Conclusion}
	    The \textbf{Facebook Network} have a different behaviour because it has \textbf{cluster} of population.
	    \vspace{0.5cm}
	    
	    The \textbf{Preferential Attachment} network is more close to the real network rather than the random network.
	    \vspace{0.5cm}
	    
	    To improve the similarity and consequently have better simulations we could implement \textbf{cluster of people} mixed with Preferential Attachment network.
	    
	    
	    
	\end{frame}
	
% 	\begin{frame}{Studio ed Implementazione: GitHub}
% 		\vspace{-0.2cm}
% 		Organizzazione delle repositories:
% 		\begin{center}
% 			\pgfuseimage{GitHubClasses}
% 		\end{center}
% 	\end{frame}
	
% 	\begin{frame}{Studio ed Implementazione: Interfaccia}
% 		\vspace{-0.2cm}
% 		Uno sguardo all'interfaccia:
% 		\begin{center}
% 			\pgfuseimage{InterfacciaGrafica}
% 		\end{center}
% 	\end{frame}
	
% 	\begin{frame}{MOSS}
% 		MOSS (Measure Of Software Similarity) è un sistema automatico per determinare la similarità di programmi.
		
% 		\begin{itemize}
% 			\item[Input] Accetta gruppi di documenti.
% 			\item[Output] Restituisce un'insieme di pagine HTML contenenti le coppie di documenti simili.
% 		\end{itemize}
		
% 		\pause
% 		Può analizzare molti linguaggi di programmazione, come ad esempio:
% 		\begin{itemize}
% 			\item \texttt{C, C++ e C\#}
% 			\item \texttt{Java}
% 			\item Il linguaggio naturale e molti altri.
% 		\end{itemize}
% 	\end{frame}

% 	\begin{frame}{MOSS: Funzionamento}
% 		Con poche fingerprint ottiene ottimi risultati. Ciò implica maggiore efficienza.
		
% 		Algoritmo:
			
% 			\vspace{-0.5cm}{\footnotesize \textbf{1)} Costruisce un indice che mappa le fingerprint alle locazioni per tutti i documenti.}
% 	\end{frame}

% 	\begin{frame}{MOSS: Funzionamento}
% 		\vspace{-0.8cm}\begin{center}
% 			\pgfuseimage{fingerprint02}
% 		\end{center}
% 		\vspace{-0.5cm}{\footnotesize \textbf{2)} Per ogni documento viene effettuata una nuova fingerprint, ottenendo una lista di fingerprint per ciascun documento. Ora ogni documento \texttt{d} può contenere fingerprint di molti altri documenti \texttt{d1,d2,...}.}
% 	\end{frame}
	
% 	\begin{frame}{MOSS: Funzionamento}
% 		\vspace{-0.5cm}\begin{center}
% 			\pgfuseimage{fingerprint03}
% 		\end{center}
% 		\vspace{-0.5cm}{\footnotesize \textbf{3)} La lista delle fingerprint viene raggruppata per documento e poi vengono fatte le coppie di documenti \texttt{(d,d1), (d,d2)}.
% 		Queste coppie vengono ordinate per numero di fingerprint uguali.}
% 	\end{frame}
	
% 	\begin{frame}{MOSS: Fingerprint}
		
% 		{\footnotesize Generare le fingerprint:
% 		\begin{itemize}[<+->]
% 			\item Divide un documento in k-grams, cioè in sottostringhe di lunghezza k
% 			\begin{exampleblock}{\footnotesize Esempio k-gram}
% 				\begin{tcolorbox}[size=small]
% 					A do run run run, a do run run
% 				\end{tcolorbox}
% 				(a) un po di testo [The Crystals. Da do run run, 1963]
% 				\pause
% 				\begin{tcolorbox}[size=small]
% 					adorunrunrunadorunrun
% 				\end{tcolorbox}
% 				(b) il testo senza alcune caratteristiche irrilevanti
% 				\pause
% 				\begin{tcolorbox}[size=small]
% 					adoru dorun orunr runru unrun nrunr runru
					
% 					unrun nruna runad unado nador adoru dorun
					
% 					orunr runru unrun
% 				\end{tcolorbox}
% 				(c) la sequenza di 5-grams derivata dal testo
% 			\end{exampleblock}
% 			\pause
% 			\item Utilizza una funzione hash per ogni k-gram e selezioniamo tra essi qualche sottoinsieme, che sarà la fingerprint.
% 		\end{itemize}
% 	}
% 	\end{frame}

% 	\begin{frame}{MOSS: Servizio}
% 		MOSS mette a disposizione un webservice accessibile tramite vari script e librerie.
		
% 		Per il nostro progetto abbiamo utilizzato la libreria Java \textbf{MOJI} che consente l'accesso al servizio.
		
% 		\pause
% 		Questa libreria permette di:
% 		\begin{enumerate}[<+->]
% 			\item Connettersi al server con il proprio numero utente.
% 			\item Inviare la cartella contenente tutti i progetti degli studenti.
% 			\item Inviare la cartella contenente il modello del progetto.
% 			\item Ricevere il link per visualizzare la pagina HTML dei risultati.
% 		\end{enumerate}
% 	\end{frame}

% 	\begin{frame}{MOSS: Servizio}
% 		\begin{block}{Struttura cartella dei progetti}
% 		 	\begin{tcolorbox}
% 				\texttt{\tab solution\_directory\\ 
% 					\tab $\vert$- student1\\
% 					\tab[1.5cm] $\vert$- main.c\\
% 					\tab[1.5cm] $\vert$- ...\\
% 					\tab $\vert$- student2\\
% 					\tab[1.5cm] $\vert$- ...\\
% 					\tab $\vert$\_ student3\\
% 					\tab[1.5cm] $\vert$- ...
% 				}
% 			\end{tcolorbox}
% 		\end{block}
% 	\end{frame}
	
% 	\begin{frame}{MOSS: Risultati}
% 		\begin{adjustwidth}{-1cm}{-1cm}
% 			\pgfuseimage{MOSSResults}
% 		\end{adjustwidth}
% 	\end{frame}



% 	\begin{frame}{MOSS: Risultati}
% 		\begin{adjustwidth}{-1cm}{-1cm}
% 			\pgfuseimage{MOSSResultMatch}
% 		\end{adjustwidth}
% 	\end{frame}

% 	\section{Conclusioni}

	
% 	\begin{frame}{Conclusioni}
% 		Obiettivi raggiunti:
% 		\begin{enumerate}
% 			\item Scaricare i progetti assegnati su GitHub Classroom.
% 			\item Gestire ed analizzare centinaia di progetti, analizzandoli tramite MOSS.
% 		\end{enumerate}
% 		L'applicazione è stata testata sul campo riportando risultati positivi.
% 		\pause
		
% 		\vspace{0.5cm}
% 		{\small \textbf{Futuri ampliamenti}:
% 		\begin{itemize}
% 			\item Download dei risultati di MOSS (MOSS li elimina dopo 14 giorni).
% 			\item Analisi statica dei progetti (es. Valgrind), per fornire un aiuto al docente con una pre-valutazione del progetto.
% 			\item Ottenere il nome e il cognome degli studenti dal file ReadMe delle repositories e costruire dinamicamente un gestionale degli strumenti.
% 		\end{itemize}}
% 	\end{frame}

	\begin{frame}{Conclusioni}
		\begin{center}
			{\Huge \textcolor{rossoPantano}{\textbf{Thanks}}}
		\end{center}
		
	\end{frame}

\end{document}

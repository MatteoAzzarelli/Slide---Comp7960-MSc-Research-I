\documentclass{beamer}
\usepackage[english,italian]{babel}

\usepackage[T1]{fontenc}
\usepackage{textcomp}
\usepackage[utf8]{inputenc}	%caratteri accentati

\newcommand\tab[1][1cm]{\hspace*{#1}}	% define new command tab

\usepackage[font=scriptsize]{subcaption}
\usepackage[most]{tcolorbox}


%------------------------------------------------
%		Trasparenza liste
%------------------------------------------------
\setbeamercovered{transparent}


%------------------------------------------------
%		IMAGE DEFINITIONS
%------------------------------------------------
\usepackage{changepage}

\usepackage{pgf,tikz}
\usepackage{graphicx}

\pgfdeclareimage[height=4cm]{GraphExample}{./images/Graph/GraphExample}
\pgfdeclareimage[height=4cm]{socialNetwork}{./images/Graph/socialNetwork}
\pgfdeclareimage[height=3.5cm]{sirModel}{./images/SIR/sir}
\pgfdeclareimage[height=2.5cm]{seirModel}{./images/SIR/SEIR-SEIRS}
\pgfdeclareimage[height=2.5cm]{sisModel}{./images/SIR/SI-SIS}
\pgfdeclareimage[height=2.5cm]{sirsModel}{./images/SIR/SIR-SIRS}

\pgfdeclareimage[width=12.8cm]{MOSSResults}{./images/MOSS/MOSSResults}
\pgfdeclareimage[width=12.8cm]{MOSSResultMatch}{./images/MOSS/MOSSResultMatch}

\pgfdeclareimage[width=10cm]{fingerprint02}{./images/MOSS/Fingerprint/fingerprint02}
\pgfdeclareimage[width=8cm]{fingerprint03}{./images/MOSS/Fingerprint/fingerprint03}

\definecolor{blueHKBU}{RGB}{0,0,210}

\usetheme{Padova}

\title{\vspace{1cm}How the spread of disease change on synthetic network and real network, simulation and comparison.}
\subtitle{\vspace{-0cm}COMP7960 Research I}
\author{%\vspace{-0.5cm}\textbf{Laureando}:\\
	Matteo Azzarelli}
\date{\vspace{2.5cm}\color{black}{\footnotesize{Academic year 2018/2019}}}



\begin{document}

    \AtBeginSection[]{
      \begin{frame}{~}
      \vfill
      \centering
      \begin{tcolorbox}[halign=center,colframe=blueHKBU, colback=blueHKBU!5!white]
        \usebeamerfont{title}\insertsectionhead\par%
      \end{tcolorbox}
      \vfill
      \end{frame}
    }
    
	\maketitle

	\begin{frame}{Outline}
		\tableofcontents
	\end{frame}

	\section{Introduction}
	
	\begin{frame}{Introduction}
		Why we developed this tool?
		\pause
		\begin{enumerate}[<+->]
			\item Evaluate the different spread behaviours of disease on different network using Python.
			\item Analyse how the spread of disease change based on different network connections using \textbf{SIR model}.
			\item Compare synthetic network with real network, Facebook like.
			
		\end{enumerate}
	\end{frame}

	\section{Graph - Network}
	
	\begin{frame}{Network}
		What is it a \textbf{Network}?
		
		\vspace{0.5cm}
		A \textbf{Network} is a particular representation of a \textit{Graph}.
		
		\vspace{0.5cm}
		A Graph is a pair of set, $G = (V, E)$, where 
		\begin{itemize}
		    \item \textbf{V} is the set of \textit{vertices} or also called \textit{
nodes}.
		    \item \textbf{E} is a set of \textit{edges}. An edge is a link between two Vertices.
		\end{itemize}
		\begin{center}
			\pgfuseimage{GraphExample}
		\end{center}
	\end{frame}
	

	\begin{frame}{Network physical meaning}
	    The fiscal meaning is:
		\begin{itemize}[<+->]
		    \item \textbf{Node}: represent the pool of person. So, each node correspond to a person in this network.
		    \item \textbf{Edges}: represent communications among people, so if there is a link between two nodes that means that they are in contact.
		    \item \textbf{Network}: is a portion of the real world that represent the connection and interaction in this mini-world.
		\end{itemize}
		\begin{left}
			\pgfuseimage{socialNetwork}
		\end{left}
	\end{frame}



	\section{SIR model}
	
	\begin{frame}{Compartmental models}
	    \textbf{Compartmental models} are a technique used to simplify the mathematical modelling of infectious disease.
	    
	    The \textbf{population} is \textbf{divided} into compartments, with the assumption that every individual in the same compartment has the \textbf{same characteristics}.
	\end{frame}
	
	\begin{frame}{SIR model}
	    What is it SIR model?
	    \begin{center}
			\pgfuseimage{sirModel}
		\end{center}
		\begin{itemize}\footnotesize
		    \item \textbf{Susceptible}: it means that this person is susceptible to ketch the disease.
		    \item \textbf{Infected}: this status represent people that have the disease and that can spread it.
		    \item \textbf{Recovered}: represent the people that had the disease and now they are recovered/immunised.
		\end{itemize}
	\end{frame}
	
	\begin{frame}{More models}
	    There are many variants of this model like:
        \begin{itemize}[<+->]
            \item \textbf{SEIRS}\only<1>{: this model provide the Exposed(E) class where the person has the disease but it doesn’t spread it on to susceptible nodes.} \only<1,4>{This kind of model is suitable for disease with a \textbf{long incubation disease}.}\only<1>{
            \begin{center}
			    \pgfuseimage{seirModel}
		    \end{center}}
            \item \textbf{SIS}\only<2>{: here there are only three states and modelling disease that after the state of infection the people come back to the state of susceptible.
            \begin{center}
			    \pgfuseimage{sisModel}
		    \end{center}
            }
            \item \textbf{SIRS}\only<3>{: this one is a variant of the previous model that introduce a stage of immunity for a certain time after that the person can be infected again.
            \begin{center}
			    \pgfuseimage{sirsModel}
		    \end{center}
            }
        \end{itemize}

        \only<4>{The previous last two models are typically used to represent different \textbf{epidemics disease} such as HIV or SARS [3].}
	\end{frame}
	
% 	\begin{frame}{Plagio}
% 		\begin{alertblock}{Definizione}
% 			Un programma che è stato prodotto da un altro e riportato con un numero esiguo di trasformazioni di routine.
			
% 			\hspace{4cm}\textit{1976. Alan Parker e James O. Hamblen.}
% 		\end{alertblock}
		
% 		\vspace{-0.5cm}
% 		\begin{center}
% 			\pgfuseimage{PlagiarismLevels}
% 		\end{center}
% 	\end{frame}
	

	\section{Network Models}
	
% 	\begin{frame}{Studio ed Implementazione}
% 		Caratteristiche dell'applicazione:
% 		\begin{itemize}
% 			\item<1-> Sviluppata in \textbf{Java}:
% 				\begin{itemize}
% 					\item \`E cross-platform.
% 					\item Documentazione completa.
% 					\item Esistenza di librerie per il nostro scopo.
% 				\end{itemize}
% 			\item<2-> Utilizzo del pattern Model View Controller.
% 			\item<3-> Download delle repositories di GitHub.
% 			\item<4-> Analisi dei progetti degli studenti tramite il servizio MOSS.
% 		\end{itemize}
% 	\end{frame}
	
	\begin{frame}{Studio ed Implementazione: MVC}
		Diagramma UML della struttura del software:
		\begin{center}
			\pgfuseimage{UMLMVC}
		\end{center}
	\end{frame}
	
	\begin{frame}{Studio ed Implementazione: GitHub}
		\vspace{-0.2cm}
		Organizzazione delle repositories:
		\begin{center}
			\pgfuseimage{GitHubClasses}
		\end{center}
	\end{frame}
	
	\begin{frame}{Studio ed Implementazione: Interfaccia}
		\vspace{-0.2cm}
		Uno sguardo all'interfaccia:
		\begin{center}
			\pgfuseimage{InterfacciaGrafica}
		\end{center}
	\end{frame}
	
	\begin{frame}{MOSS}
		MOSS (Measure Of Software Similarity) è un sistema automatico per determinare la similarità di programmi.
		
		\begin{itemize}
			\item[Input] Accetta gruppi di documenti.
			\item[Output] Restituisce un'insieme di pagine HTML contenenti le coppie di documenti simili.
		\end{itemize}
		
		\pause
		Può analizzare molti linguaggi di programmazione, come ad esempio:
		\begin{itemize}
			\item \texttt{C, C++ e C\#}
			\item \texttt{Java}
			\item Il linguaggio naturale e molti altri.
		\end{itemize}
	\end{frame}

	\begin{frame}{MOSS: Funzionamento}
		Con poche fingerprint ottiene ottimi risultati. Ciò implica maggiore efficienza.
		
		Algoritmo:
			
			\vspace{-0.5cm}{\footnotesize \textbf{1)} Costruisce un indice che mappa le fingerprint alle locazioni per tutti i documenti.}
	\end{frame}

	\begin{frame}{MOSS: Funzionamento}
		\vspace{-0.8cm}\begin{center}
			\pgfuseimage{fingerprint02}
		\end{center}
		\vspace{-0.5cm}{\footnotesize \textbf{2)} Per ogni documento viene effettuata una nuova fingerprint, ottenendo una lista di fingerprint per ciascun documento. Ora ogni documento \texttt{d} può contenere fingerprint di molti altri documenti \texttt{d1,d2,...}.}
	\end{frame}
	
	\begin{frame}{MOSS: Funzionamento}
		\vspace{-0.5cm}\begin{center}
			\pgfuseimage{fingerprint03}
		\end{center}
		\vspace{-0.5cm}{\footnotesize \textbf{3)} La lista delle fingerprint viene raggruppata per documento e poi vengono fatte le coppie di documenti \texttt{(d,d1), (d,d2)}.
		Queste coppie vengono ordinate per numero di fingerprint uguali.}
	\end{frame}
	
	\begin{frame}{MOSS: Fingerprint}
		
		{\footnotesize Generare le fingerprint:
		\begin{itemize}[<+->]
			\item Divide un documento in k-grams, cioè in sottostringhe di lunghezza k
			\begin{exampleblock}{\footnotesize Esempio k-gram}
				\begin{tcolorbox}[size=small]
					A do run run run, a do run run
				\end{tcolorbox}
				(a) un po di testo [The Crystals. Da do run run, 1963]
				\pause
				\begin{tcolorbox}[size=small]
					adorunrunrunadorunrun
				\end{tcolorbox}
				(b) il testo senza alcune caratteristiche irrilevanti
				\pause
				\begin{tcolorbox}[size=small]
					adoru dorun orunr runru unrun nrunr runru
					
					unrun nruna runad unado nador adoru dorun
					
					orunr runru unrun
				\end{tcolorbox}
				(c) la sequenza di 5-grams derivata dal testo
			\end{exampleblock}
			\pause
			\item Utilizza una funzione hash per ogni k-gram e selezioniamo tra essi qualche sottoinsieme, che sarà la fingerprint.
		\end{itemize}
	}
	\end{frame}

	\begin{frame}{MOSS: Servizio}
		MOSS mette a disposizione un webservice accessibile tramite vari script e librerie.
		
		Per il nostro progetto abbiamo utilizzato la libreria Java \textbf{MOJI} che consente l'accesso al servizio.
		
		\pause
		Questa libreria permette di:
		\begin{enumerate}[<+->]
			\item Connettersi al server con il proprio numero utente.
			\item Inviare la cartella contenente tutti i progetti degli studenti.
			\item Inviare la cartella contenente il modello del progetto.
			\item Ricevere il link per visualizzare la pagina HTML dei risultati.
		\end{enumerate}
	\end{frame}

	\begin{frame}{MOSS: Servizio}
		\begin{block}{Struttura cartella dei progetti}
		 	\begin{tcolorbox}
				\texttt{\tab solution\_directory\\ 
					\tab $\vert$- student1\\
					\tab[1.5cm] $\vert$- main.c\\
					\tab[1.5cm] $\vert$- ...\\
					\tab $\vert$- student2\\
					\tab[1.5cm] $\vert$- ...\\
					\tab $\vert$\_ student3\\
					\tab[1.5cm] $\vert$- ...
				}
			\end{tcolorbox}
		\end{block}
	\end{frame}
	
	\begin{frame}{MOSS: Risultati}
		\begin{adjustwidth}{-1cm}{-1cm}
			\pgfuseimage{MOSSResults}
		\end{adjustwidth}
	\end{frame}



	\begin{frame}{MOSS: Risultati}
		\begin{adjustwidth}{-1cm}{-1cm}
			\pgfuseimage{MOSSResultMatch}
		\end{adjustwidth}
	\end{frame}

	\section{Conclusioni}

	
	\begin{frame}{Conclusioni}
		Obiettivi raggiunti:
		\begin{enumerate}
			\item Scaricare i progetti assegnati su GitHub Classroom.
			\item Gestire ed analizzare centinaia di progetti, analizzandoli tramite MOSS.
		\end{enumerate}
		L'applicazione è stata testata sul campo riportando risultati positivi.
		\pause
		
		\vspace{0.5cm}
		{\small \textbf{Futuri ampliamenti}:
		\begin{itemize}
			\item Download dei risultati di MOSS (MOSS li elimina dopo 14 giorni).
			\item Analisi statica dei progetti (es. Valgrind), per fornire un aiuto al docente con una pre-valutazione del progetto.
			\item Ottenere il nome e il cognome degli studenti dal file ReadMe delle repositories e costruire dinamicamente un gestionale degli strumenti.
		\end{itemize}}
	\end{frame}

	\begin{frame}{Conclusioni}
		\begin{center}
			{\Huge \textcolor{rossoPantano}{\textbf{Grazie per l'attenzione}}}
		\end{center}
		
	\end{frame}

\end{document}
